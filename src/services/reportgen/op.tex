
\documentclass{article}
\usepackage{graphicx}
\usepackage{listings}

\begin{document}



\centering
\textbf{\Huge Lab Record}
\\[1in]


\includegraphics[height=250pt]{static/rvlogo.jpg}
\\[1in]

{\huge
     \\~\\
    Incharge: } \\~\\

\pagebreak


\textbf{\Huge Certificate}\\[1in]

\begin{flushleft}
{\large
    This is to certify that Mr. Divyanshu Kakwani, USN 1RV13IS015 of Semester 6, Information Science and Engineering has
satisfactorily completed the course of experiments in practical Database Management system
prescribed by VTU university during the academic year 2015-2016.\\[1in]}
\end{flushleft}


\begin{flushleft}
{\large
    Name: Divyanshu Kakwani \\~\\
    USN: 1RV13IS015 \\~\\
}
\end{flushleft}
\pagebreak


\pagebreak



{\large Semaphore implementation  } \\[0.5in]
\begin{flushleft}
\begin{lstlisting}
    Specification: implement sema_up and sema_down
\end{lstlisting}\\[1in]
\end{flushleft}


\begin{lstlisting}
        {
  "id": "dfada900af9b11e5beecf94377859ff1",
  "fontType": "Helvetica Neue",
  "version": "1.2.1",
  "paneVertical": "39.75%",
  "paneHorizontalRight": "51.838235294117645%",
  "paneHorizontalLeft": "51.838235294117645%",
  "pythonCmd": "/usr/bin/python3",
  "theme": "css/styles.css"
}
    \end{lstlisting}
\pagebreak


{\large TCP Server  } \\[0.5in]
\begin{flushleft}
\begin{lstlisting}
    Task: Create a tcp server
\end{lstlisting}\\[1in]
\end{flushleft}


\begin{lstlisting}
        LAST_EPOCH=16918

    \end{lstlisting}
\pagebreak


{\large Hasmukh and his birthday party  } \\[0.5in]
\begin{flushleft}
\begin{lstlisting}
    

### Problem



Hasmukh is celebrating his 20th birthday in his hometown Jhalor. To make his birthday party grand,

he has decided to invite all the people of Jhalor to his party.



The city Jhalor consists of N houses interconnected by exactly N-1 roads in such a way that any house can be reached

by any other. One of these houses is owned by Hasmukh from where he sends out

his servants, one to each house of the city, to distribute the invitation cards. 



As commanded by Hasmukh, each servant goes directly to his assigned house as quickly as he can, drops the invitation instantly, and then returns to Hasmukh's house, again as quickly as he can.



Hasmukh is tight on time. So he wants you to tell him how long would it take for all of his

servants to return home so that he can give them further errands.



### Input



The first line contains the number of test cases T. Each test case starts with a line containing

N, the number of vertices, and H, the location of Hasmukh's house. This is followed by N-1 lines each 

containing three numbers - u, v, D - where u and v are the two ends of a road and D is the transit time(in mins)

through this road.



Note that the houses are numbered from 0 to N-1.



### Output



For each test case, print the time taken, in mins, for all the servants to return home.



### Constraints

```

0 < T <= 100

0 < N <= 1000

0 < H < N

0 < u, v < N

0 < D <= 10^9

```



### Time Limit

`1.0000 seconds`



### Example



Input

```

1

3 0

0 1 4

0 2 5

```



Output

```

10

```

Explanation



All the servants start at house 0. Following Hasmukh's command, one of the servants goes to

the house numbered 1 and another to the house numbered 2. The first one returns in 8 mins,

4 to reach house 1 and 4 to return, and the second returns in 10 mins. Hence, the answer is 10.


\end{lstlisting}\\[1in]
\end{flushleft}


\begin{lstlisting}
    \end{lstlisting}
\pagebreak




\end{document}
